\documentclass[hyperref={pdfpagelabels=false}]{beamer}
\usetheme{Boadilla}

\usepackage[T1]{fontenc}
\usepackage[brazil]{babel}
\usepackage[utf8]{inputenc}

\usepackage[cache=false]{minted}

\usepackage{pgfpages}
\usepackage{listings}

\usepackage{amsfonts}
\usepackage{amsmath}
\usepackage{amssymb}
\usepackage{graphicx}
\usepackage{mathptmx}
\usepackage{lmodern}

\usepackage{hyperref}

\usepackage{tcolorbox}
\usepackage{xcolor}
\usepackage{color}


\usepackage{DejaVuSansMono}

\beamertemplatenavigationsymbolsempty

\tcbuselibrary{breakable}
\tcbuselibrary{listings}
\tcbuselibrary{minted}
\tcbuselibrary{skins}

\definecolor{codebg}{RGB}{250,250,250} % kate
\definecolor{linenosbg}{RGB}{220,220,220}
\definecolor{monokaibg}{HTML}{282828} % monokai

\renewcommand*\familydefault{\ttdefault}
\renewcommand\listoflistingscaption{List of source codes}
\renewcommand\theFancyVerbLine{\normal\arabic{FancyVerbLine}}

\newtcblisting{python3}{
    listing engine=minted,
    listing only,
    minted style=kate,
    minted language=python3,
    minted options={fontsize=\small,linenos,numbersep=3mm},
    colback=codebg,
    colframe=codebg,
    left=6mm,
    enhanced,
    overlay={\begin{tcbclipinterior}\fill[linenosbg](frame.south
    west)rectangle([xshift=5mm]frame.north west) ;\end{tcbclipinterior}}
}

\newtcblisting{bash}{
    listing engine=minted,
    listing only,
    minted style=monokai,
    minted language=bash,
    minted options={fontsize=\small,linenos,numbersep=3mm},
    colback=monokaibg,
    colframe=monokaibg,
    left=6mm,
    enhanced,
    overlay={\begin{tcbclipinterior}\fill[linenosbg](frame.south
    west)rectangle([xshift=5mm]frame.north west) ;\end{tcbclipinterior}}
}

\title[Python Easy]{
    {Hello, scientists!}\\[10mm]
    Would scientist use python to do your own academic research?
}

\author[Prof.~Iury Adones]{
    Prof.~Iury Adones Xavier dos Santos
}

\institute[UFRPE]{
    Rural Federal University of Pernambuco
}

\date[Brazil, Recife--PE]{}

\begin{document}

\begin{frame}
  \maketitle
\end{frame}

\begin{frame}{Python Collections (Arrays)}

There are four collection data types in the Python programming language:

\begin{itemize}[<+->]
\item \textbf{List}
    \begin{itemize}
    \item Is a collection which is \textbf{ordered} and \textbf{changeable}.
    \item \textbf{Allows duplicate} members.
    \end{itemize}
\item \textbf{Tuple}
    \begin{itemize}
    \item Is a collection which is \textbf{ordered} and \textbf{unchangeable}.
    \item \textbf{Allows duplicate} members.
    \end{itemize}
\end{itemize}
\end{frame}

\begin{frame}{Python Collections (Arrays)}

There are four collection data types in the Python programming language:

\begin{itemize}[<+->]
\item \textbf{Set}
    \begin{itemize}
    \item Is a collection which is \textbf{unordered} and \textbf{unindexed}.
    \item \textbf{No duplicate} members.
    \end{itemize}
\item \textbf{Dictionary}
    \begin{itemize}
    \item Is a collection which is \textbf{unordered}, \textbf{changeable} and \textbf{indexed}.
    \item \textbf{No duplicate} members.
    \end{itemize}
\end{itemize}
\end{frame}

\begin{frame}{About List}
\centering
\Huge{list()}
\end{frame}

\begin{frame}[fragile]{List}
\begin{exampleblock}{Create a List:}
\begin{python3}
this_list = ["apple", "banana", "cherry"]
print(this_list)
\end{python3}
\end{exampleblock}
\end{frame}

\begin{frame}[fragile]{Access Items}
\begin{exampleblock}{Print the second item of the list:}
\begin{python3}
this_list = ["apple", "banana", "cherry"]
print(this_list[1])
\end{python3}
\end{exampleblock}
\end{frame}

\begin{frame}[fragile]{Change Item Value}
\begin{exampleblock}{Change the second item:}
\begin{python3}
this_list = ["apple", "banana", "cherry"]
this_list[1] = "blackcurrant"
print(this_list)
\end{python3}
\end{exampleblock}
\end{frame}

\begin{frame}[fragile]{List Length}
\begin{exampleblock}{Print the number of items in the list:}
\begin{python3}
this_list = ["apple", "banana", "cherry"]
print(len(this_list))
\end{python3}
\end{exampleblock}
\end{frame}

\begin{frame}[fragile]{Add Items}
\begin{exampleblock}{Using the \textbf{append()} method to append an item:}
\begin{python3}
this_list = ["apple", "banana", "cherry"]
this_list.append("orange")
print(this_list)
\end{python3}
\end{exampleblock}
\end{frame}

\begin{frame}[fragile]{Add Items}
\begin{exampleblock}{Insert an item as the second position:}
\begin{python3}
this_list = ["apple", "banana", "cherry"]
this_list.insert(1, "orange")
print(this_list)
\end{python3}
\end{exampleblock}
\end{frame}

\begin{frame}[fragile]{Remove Item}
\begin{exampleblock}{The textbf{remove()} method removes the specified item:}
\begin{python3}
this_list = ["apple", "banana", "cherry"]
this_list.remove("banana")
print(this_list)
\end{python3}
\end{exampleblock}
\end{frame}

\begin{frame}[fragile]{Remove Item}
\begin{exampleblock}{The \textbf{pop()} method removes the specified index, (or the last item if index is not specified):}
\begin{python3}
this_list = ["apple", "banana", "cherry"]
this_list.pop()
print(this_list)
\end{python3}
\end{exampleblock}
\end{frame}

\begin{frame}[fragile]{Remove Item}
\begin{exampleblock}{The \textbf{del} keyword removes the specified index:}
\begin{python3}
this_list = ["apple", "banana", "cherry"]
del this_list[0]
print(this_list)
\end{python3}
\end{exampleblock}
\end{frame}

\begin{frame}[fragile]{Remove Item}
\begin{alertblock}{The \textbf{del} keyword can also delete the list completely:}
\begin{python3}
this_list = ["apple", "banana", "cherry"]
del this_list
print(this_list)
\end{python3}
\end{alertblock}
\end{frame}

\begin{frame}[fragile]{Remove Item}
\begin{exampleblock}{The {\color{red}clear()} method empties the list:}
\begin{python3}
this_list = ["apple", "banana", "cherry"]
this_list.clear()
print(this_list)
\end{python3}
\end{exampleblock}
\end{frame}

\begin{frame}[fragile]{The list() Constructor}
\begin{exampleblock}{Using the list() constructor to make a List:}
\begin{python3}
this_list = list(("apple", "banana", "cherry"))
print(this_list)
\end{python3}
\end{exampleblock}

Note the double round-brackets
\end{frame}

\begin{frame}[fragile]{List Methods}
Python has a set of built-in methods that you can use on lists.
\begin{itemize}[<+->]
\item How to get all the methods of \textbf{list()}?
\item With \textbf{dir()}
\end{itemize}

\end{frame}

\begin{frame}[fragile]{List Methods}
\begin{exampleblock}{Use \textbf{dir()}:}
\begin{python3}
list_all_methods = dir(list)
print(list_all_methods)
\end{python3}
\end{exampleblock}
\end{frame}

\begin{frame}[fragile]{List Methods}
\begin{exampleblock}{Look up!}
\begin{python3}
list.append(self, value)
list.clear(self)
list.insert(self, index, value)
list.pop(self, index=-1)
list.remove(self, value)
\end{python3}
\end{exampleblock}
\end{frame}

\begin{frame}[fragile]{List Methods}
\begin{exampleblock}{More on Lists}
\begin{python3}
list.copy(self)
list.count(self, value)
list.extend(self, iterable)
list.index(self, value, start=0, stop=len(self))
list.reverse(self)
list.sort(self, key=None, reverse=False)
\end{python3}
\end{exampleblock}
\end{frame}


\begin{frame}[fragile]{Method copy()}
\begin{exampleblock}{Copy}
\begin{python3}
fruits = ['orange', 'apple', 'pear',
          'banana', 'kiwi', 'apple',
          'banana']

a_fruits = fruits
b_fruits = fruits.copy()

id_a = id(a_fruits)
id_b = id(b_fruits)
id_c = id(fruits)

print(id_a == id_c)
print(id_b == id_c)
\end{python3}
\end{exampleblock}
\end{frame}

\begin{frame}[fragile]{Methods count() and index()}
\begin{exampleblock}{Count and Index}
\begin{python3}
fruits = ['orange', 'apple', 'pear',
          'banana', 'kiwi', 'apple',
          'banana']

print(fruits.count('apple'))
print(fruits.count('tangerine'))

print(fruits.index('banana'))
print(fruits.index('banana', 4))
\end{python3}
\end{exampleblock}
\end{frame}

\begin{frame}[fragile]{Methods reverse() and sort()}
\begin{exampleblock}{Copy}
\begin{python3}
fruits = ['orange', 'apple', 'pear',
          'banana', 'kiwi', 'apple',
          'banana']

fruits.reverse()
print(fruits)

fruits.sort()
print(fruits)

fruits.sort(reverse=True)
print(fruits)
\end{python3}
\end{exampleblock}
\end{frame}

\begin{frame}[fragile]{Methods sort()}
\begin{exampleblock}{Sort}
\begin{python3}
fruits = ['orange', 'apple', 'pear',
          'banana', 'kiwi', 'apple',
          'banana']

fruits.sort(key=lambda f: f[-1])
print(fruits)

fruits.sort(key=lambda f: f[2])
print(fruits)
\end{python3}
\end{exampleblock}
\end{frame}

\begin{frame}[fragile]{Methods extend()}
\begin{exampleblock}{Extend}
\begin{python3}
fruits = ['orange', 'apple', 'pear']
add_fruits = ['banana', 'kiwi', 'apple', 'banana']

fruits.append(add_fruits)
print(fruits)

fruits.pop()

fruits.extend(add_fruits)
print(fruits)
\end{python3}
\end{exampleblock}
\end{frame}

\begin{frame}[fragile]{Methods built-ins max() and min()}
\begin{exampleblock}{Max and Min}
\begin{python3}
points = [5.3, 6, 7. 8, 10]

max_point = max(points)
min_point = min(points)

print(f"Max: {max_point}")
print(f"Min: {min_point}")
\end{python3}
\end{exampleblock}
\end{frame}

\begin{frame}[fragile]{Loop Through a List}
\begin{exampleblock}{Print all items in the list, one by one:}
\begin{python3}
this_list = ["apple", "banana", "cherry"]
for x in this_list:
    print(x)
\end{python3}
\end{exampleblock}
\end{frame}

\begin{frame}[fragile] \begin{block}{Pratice}
Make a Program that asks for the \textbf{\textit{4 bimonthly notes}} and
\textbf{\textit{adds to a list}}, then calculate and print to show the
information of \textbf{mean}, \textbf{median}, \textbf{standard deviation},
\textbf{maximum note} and \textbf{minimum note}.\\[5mm]

Mean: $ \bar{x} = \frac{1}{n}\sum_{i=0}^{n-1} x_{i} $\\[5mm]

Standard deviation: $$ \sigma = \sqrt{\frac{1}{n}\sum_{i=0}^{n-1}(x_{i} - \bar{x})^{2}} $$
Standard deviation relative to sample: $$ s = \sqrt{\frac{1}{n-1}\sum_{i=0}^{n-1}(x_{i} - \bar{x})^{2}} $$
\end{block}

\end{frame}

\begin{frame}{About tuple}
\centering
\Huge{tuple()}
\end{frame}

\begin{frame}[fragile]{Tuple}
\begin{exampleblock}{Create a Tuple:}
\begin{python3}
this_tuple = ("apple", "banana", "cherry")
print(this_tuple)
\end{python3}
\end{exampleblock}
\end{frame}

\begin{frame}[fragile]{Access Tuple Items}
\begin{exampleblock}{Return the item in position 1:}
\begin{python3}
this_tuple = ("apple", "banana", "cherry")
print(this_tuple[1])
\end{python3}
\end{exampleblock}
\end{frame}

\begin{frame}[fragile]{Tuple Length}
\begin{exampleblock}{Print the number of items in the tuple:}
\begin{python3}
this_tuple = ("apple", "banana", "cherry")
print(len(this_tuple))
\end{python3}
\end{exampleblock}
\end{frame}

\begin{frame}[fragile]{Remove Items}
\begin{exampleblock}{The del keyword can delete the tuple completely:}
\begin{python3}
this_tuple = ("apple", "banana", "cherry")
del this_tuple
\end{python3}
\end{exampleblock}
\end{frame}

\begin{frame}[fragile]{The tuple() Constructor}
\begin{exampleblock}{Using the tuple() method to make a tuple:}
\begin{python3}
this_tuple = tuple(("apple", "banana", "cherry"))
print(this_tuple)
\end{python3}
\end{exampleblock}

Note the double round-brackets
\end{frame}

\begin{frame}[fragile]{Tuple Methods}
Python has a set of built-in methods that you can use on tuples.
\begin{itemize}[<+->]
\item How to get all the methods of \textbf{tuple()}?
\item With \textbf{dir()}
\end{itemize}

\end{frame}

\begin{frame}[fragile]{Tuple Methods}
\begin{exampleblock}{Use \textbf{dir()}:}
\begin{python3}
tuple_all_methods = dir(tuple)
print(tuple_all_methods)
\end{python3}
\end{exampleblock}
\end{frame}

\begin{frame}[fragile]{Tuple Methods}
\begin{exampleblock}{More on tuples}
\begin{python3}
tuple.count(self, value)
tuple.index(self, value, start=0, stop=len(self))
\end{python3}
\end{exampleblock}
\end{frame}

\begin{frame}{The end}
\centering
\Large{Thant's all Folks!}
\end{frame}

\end{document}

% \begin{frame}[fragile]{}
% \begin{exampleblock}{}
% \begin{python3}

% \end{python3}
% \end{exampleblock}
% \end{frame}








% \begin{frame}
% Each frame should have a title.
% \end{frame}

% \begin{frame}
% Without title somethink is missing. 
% \end{frame}

% \begin{frame}
% \frametitle{unnumbered lists}
% \begin{itemize}
% \item Introduction to  \LaTeX{}  
% \item Course 2 
% \item Termpapers and presentations with \LaTeX{}  
% \item Beamer class
% \end{itemize} 
% \end{frame}

% \begin{frame}\frametitle{lists with single pauses}
% \begin{itemize}
% \item Introduction to  \LaTeX{}  \pause 
% \item Course 2 \pause 
% \item Termpapers and presentations with \LaTeX{}  \pause 
% \item Beamer class
% \end{itemize} 
% \end{frame}

% \begin{frame}\frametitle{lists with pause}
% \begin{itemize}[<+->]
% \item Introduction to  \LaTeX{}  
% \item Course 2
% \item Termpapers and presentations with \LaTeX{}  
% \item Beamer class
% \end{itemize} 
% \end{frame}

% \begin{frame}\frametitle{numbered lists}
% \begin{enumerate}
% \item Introduction to  \LaTeX{}   
% \item Course 2 
% \item Termpapers and presentations with \LaTeX{}  
% \item Beamer class
% \end{enumerate}
% \end{frame}

% \begin{frame}
% \frametitle{numbered lists with single pauses}
% \begin{enumerate}
% \item Introduction to  \LaTeX{}  \pause 
% \item Course 2 \pause 
% \item Termpapers and presentations with \LaTeX{}  \pause 
% \item Beamer class
% \end{enumerate}
% \end{frame}

% \begin{frame}
% \frametitle{numbered lists with pause}
% \begin{enumerate}[<+->]
% \item Introduction to  \LaTeX{}  
% \item Course 2
% \item Termpapers and presentations with \LaTeX{}  
% \item Beamer class
% \end{enumerate}
% \end{frame}

% \begin{frame}
% \frametitle{Tables}
% \begin{tabular}{|c|c|c|}
% \hline
% \textbf{Date} & \textbf{Instructor} & \textbf{Title} \\
% \hline
% WS 04/05 & Sascha Frank & First steps with  \LaTeX  \\
% \hline
% SS 05 & Sascha Frank & \LaTeX \ Course serial \\
% \hline
% \end{tabular}
% \end{frame}

% \begin{frame}
% \frametitle{Tables with pause}
% \begin{tabular}{c c c}
% A & B & C \\ 
% \pause 
% 1 & 2 & 3 \\  
% \pause 
% A & B & C \\ 
% \end{tabular} 
% \end{frame}

% \begin{frame}
% \frametitle{blocs}

% \begin{block}{title of the bloc}
% bloc text
% \end{block}

% \begin{exampleblock}{title of the bloc}
% bloc text
% \end{exampleblock}


% \begin{alertblock}{title of the bloc}
% bloc text
% \end{alertblock}
% \end{frame}

% \begin{frame}[fragile]{Example: code of {\color{cyan}\textbf{python3}}}
% \begin{python3}
% class Person:
%     def __init__(self, name=None):
%         self.name = f"{name}"


% def main():
%     person = Person()
%     print(f"hello, {person.name}")

%     person.name = "John"
%     print(f"hello, {person.name}")


% if __name__ == "__main__":
%     main()
% \end{python3}
% \end{frame}

% \begin{frame}[fragile]{Example: \textbf{echo} on {\color{magenta}\textbf{bash}}}
% \begin{bash}
% echo "hello, world!"
% \end{bash}
% \end{frame}
